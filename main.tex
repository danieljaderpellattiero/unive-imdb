% Preamble
\documentclass[12pt, twoside]{report}

% Packages
\usepackage[utf8]{inputenc}
\usepackage{amsmath}
\usepackage{csquotes}
\usepackage{graphicx}
\usepackage[english]{babel}
\usepackage[a4paper, top=1in, bottom=0.8in, left=1in, right=1in]{geometry}
\usepackage{fancyhdr}
\usepackage{comment}

% Graphics path
\graphicspath{{images/}}

\begin{document}
	\pagenumbering{roman}

	% Title Page
	\begin{titlepage}

		\renewcommand{\thesection}{\arabic{section}}
		\newcommand{\HRule}{\rule{\linewidth}{0.2mm}}
		
		\center
		
		%----------------------------------------------------------------------------------------
		%	HEADING SECTIONS
		%----------------------------------------------------------------------------------------
		
		\textsc{\Large Ca' Foscari University of Venice }\\[1.0cm]
		\includegraphics[scale=1]{ unive-emblem.png }\\[1cm]
		\textsc{\Large Department of Environmental Sciences, Informatics and Statistics }\\[0.5cm]
		\textsc{\large MCS Computer Science and Information Technology CM-90 }\\[0.5cm]
		\textsc{\large [CM0481] Software Performance and Scalability }\\[0.5cm]
		
		%----------------------------------------------------------------------------------------
		%	TITLE SECTION
		%----------------------------------------------------------------------------------------
		
		\HRule \\[0.8cm]
		{ \Huge \bfseries IMDb: Performance and Scalability Analysis }\\[0.5cm]
		\HRule \\[0.8cm]
		
		%----------------------------------------------------------------------------------------
		%	AUTHOR SECTION
		%----------------------------------------------------------------------------------------
		
		\Large \emph{Authors:}\\
		Michele \textsc{Lotto} - {\large 875922}\\
		Daniel Jader \textsc{Pellattiero} - {\large 903837}\\[3.5cm]
		
		%----------------------------------------------------------------------------------------
		%	DATE SECTION
		%----------------------------------------------------------------------------------------
		
		{\large Rev. 26/06/2024  }\\
		
		\vfill
		
	\end{titlepage}

	% Table of Contents, List of Figures, List of Tables
	\tableofcontents
	\listoftables
	\addcontentsline{toc}{chapter}{List of Tables}
	\listoffigures
	\addcontentsline{toc}{chapter}{List of Figures}

	% Abstract
	\chapter*{Abstract}
	\addcontentsline{toc}{chapter}{Abstract}
	% Descrizione consegna del progetto
	% This report presents a comprehensive account of the implementation and performance testing process conducted on a web application that emulates the IMDb website\footnote[1]{https://www.imdb.com/}.

The report begins with an overview of the system implementation, in which the technologies utilised are outlined. Subsequently, a theoretical analysis of the queueing system network is presented, in which the implemented system is characterised.
The load test conducted on the system will be subjected to a thorough examination, including an analysis of its theoretical and empirical findings.

In conclusion, a potential architectural solution is evaluated for its potential to enhance the scalability of the tested system.

It should be noted that the implementation of the system, along with all automated programs and scripts referenced in this report, is fully documented and accessible on the GitHub repository\footnote[2]{https://github.com/danieljaderpellattiero/unive-imdb}.


	% Body of the paper
	\chapter{Introduction}
	\pagenumbering{arabic}
	\setcounter{page}{1}

	% Introduction chapter
	% \input{chapters/introduction}

	% Other chapters
	% \chapter{Chapter Title}
	% \input{chapters/chapter01}

	% Conclusion chapter
	% \chapter{Conclusion}
	% \input{chapters/conclusion}

	% Bibliography
	% \bibliographystyle{IEEEtran}
	% \bibliography{IEEEexample}

	\begin{comment}
	Introduzione:
		- Descrizione alto livello del sistema.
	Capitolo 1: Implementazione del sistema
		- Database.
			- Data analysis. (sanitization, rimozione film senza rating, guarda lo script)
			- Schema. (scorporamento degli episodi dai titoli, guarda schema PDF)
			- Indici + bulk loading e automatization.
		- API.
			- Endpoints.
		- Frontend.
	Capitolo 2: Test case identification.
		- Spiegazione a parole + grafo.
	Capitolo 3: Theoretical queueing network
		- Queueing network components.
		- Diagramma traffic equations.
		- Traffic equations. (with solutions)
	Capitolo 4: Theoretical queueing network analysis
		- Service times measurements.
		- Bottleneck identification.
		- Mean Value Analysis.
			- Throughput over # users. (+ graphic)
			- Response time over # users. (+ graphic)
		- Asymptitic bounds and optimal number of users.
			- Bounds calculation.
			- Throughput over # users with bounds. (+ graphic)
			- Response time over # users with bounds. (+ graphic)
			- Optimal number of users.
	Capitolo 5: Experimental network analysis
		- Illustrazione load test JMeter.
		- Spiegazione configurazione del load test.
		- Grafici throughput e response time arricchiti.
	Capitolo 6: Results comparison
		- Confronto analisi teorica e sperimentale.
	Capitolo 7: Conclusioni
		- Proposta di architettura scalabile.
			- MongoDB caching.
			- MongoDB Sharding e replica.
			- Analisi prestazioni sistema proposto con MVA.
	\end{comment}

\end{document}
