This brief chapter aims to present the test case designed to perform the load test of the system.
In order to illustrate the user behaviour modelled by the test, we define a trivial finite-state automata with reference to the endpoints implemented in the RESTful API.

\begin{center}
	\begin{tikzpicture}[shorten >=1pt,node distance=2cm,on grid,auto]
		\tikzstyle{every state}=[fill={rgb:black,1;white,10}]
		\node[state, initial, accepting] (0) {$T$};
		\node[state] (1) [right of = 0] {$S_0$};
		\node[state] (2) [below of = 1] {$S_1$};
		\path[->]
		(0) edge node {$p_{1}$} (1)
		(1) edge [loop above] node {$p_{0.1}$} (1)
		(1) edge [bend left] node {$p_{0.1}$} (0)
		(1) edge node {$p_{0.8}$} (2)
		(2) edge [bend left] node {$p_{1}$} (0);
		;
	\end{tikzpicture}
\end{center}

The above automata has to be interpreted in the following manner:

\begin{itemize}
	\item The user initiates a title search by name match, regardless of the customer language locale. (\verb|search/:title|)
	\begin{itemize}
		\item In $80\%$ of cases, the search continues with a precise request for the title information. (\verb|/title/:id| \textit{or} \verb|/episode/:id|)
		\item In $20\%$ of cases, either the user terminates the search process pre-emptively, without finding the title, thus entering a dormant (thinking) state; or it continues the search by hitting the same endpoint though the use of pagination. (\verb|search/:title?page=n|)
	\end{itemize}
	\item Upon receipt of the searched title information, the user is satisfied and returns to a dormant (thinking) state before repeating a new search.
\end{itemize}